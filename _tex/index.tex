%%
% Copyright (c) 2017 - 2024, Pascal Wagler;
% Copyright (c) 2014 - 2024, John MacFarlane
%
% All rights reserved.
%
% Redistribution and use in source and binary forms, with or without
% modification, are permitted provided that the following conditions
% are met:
%
% - Redistributions of source code must retain the above copyright
% notice, this list of conditions and the following disclaimer.
%
% - Redistributions in binary form must reproduce the above copyright
% notice, this list of conditions and the following disclaimer in the
% documentation and/or other materials provided with the distribution.
%
% - Neither the name of John MacFarlane nor the names of other
% contributors may be used to endorse or promote products derived
% from this software without specific prior written permission.
%
% THIS SOFTWARE IS PROVIDED BY THE COPYRIGHT HOLDERS AND CONTRIBUTORS
% "AS IS" AND ANY EXPRESS OR IMPLIED WARRANTIES, INCLUDING, BUT NOT
% LIMITED TO, THE IMPLIED WARRANTIES OF MERCHANTABILITY AND FITNESS
% FOR A PARTICULAR PURPOSE ARE DISCLAIMED. IN NO EVENT SHALL THE
% COPYRIGHT OWNER OR CONTRIBUTORS BE LIABLE FOR ANY DIRECT, INDIRECT,
% INCIDENTAL, SPECIAL, EXEMPLARY, OR CONSEQUENTIAL DAMAGES (INCLUDING,
% BUT NOT LIMITED TO, PROCUREMENT OF SUBSTITUTE GOODS OR SERVICES;
% LOSS OF USE, DATA, OR PROFITS; OR BUSINESS INTERRUPTION) HOWEVER
% CAUSED AND ON ANY THEORY OF LIABILITY, WHETHER IN CONTRACT, STRICT
% LIABILITY, OR TORT (INCLUDING NEGLIGENCE OR OTHERWISE) ARISING IN
% ANY WAY OUT OF THE USE OF THIS SOFTWARE, EVEN IF ADVISED OF THE
% POSSIBILITY OF SUCH DAMAGE.
%%

%%
% This is the Eisvogel pandoc LaTeX template.
%
% For usage information and examples visit the official GitHub page:
% https://github.com/Wandmalfarbe/pandoc-latex-template
%%

% Options for packages loaded elsewhere
\PassOptionsToPackage{unicode}{hyperref}
\PassOptionsToPackage{hyphens}{url}
\PassOptionsToPackage{dvipsnames,svgnames,x11names,table}{xcolor}
%
\documentclass[
  paper=a4,
  twoside  ,captions=tableheading
]{scrbook}
\usepackage{amsmath,amssymb}
% Use setspace anyway because we change the default line spacing.
% The spacing is changed early to affect the titlepage and the TOC.
\usepackage{setspace}
\setstretch{1.2}
\usepackage{iftex}
\ifPDFTeX
  \usepackage[T1]{fontenc}
  \usepackage[utf8]{inputenc}
  \usepackage{textcomp} % provide euro and other symbols
\else % if luatex or xetex
  \usepackage{unicode-math} % this also loads fontspec
  \defaultfontfeatures{Scale=MatchLowercase}
  \defaultfontfeatures[\rmfamily]{Ligatures=TeX,Scale=1}
\fi
\usepackage{lmodern}
\ifPDFTeX\else
  % xetex/luatex font selection
\fi
% Use upquote if available, for straight quotes in verbatim environments
\IfFileExists{upquote.sty}{\usepackage{upquote}}{}
\IfFileExists{microtype.sty}{% use microtype if available
  \usepackage[]{microtype}
  \UseMicrotypeSet[protrusion]{basicmath} % disable protrusion for tt fonts
}{}
\makeatletter
\@ifundefined{KOMAClassName}{% if non-KOMA class
  \IfFileExists{parskip.sty}{%
    \usepackage{parskip}
  }{% else
    \setlength{\parindent}{0pt}
    \setlength{\parskip}{6pt plus 2pt minus 1pt}}
}{% if KOMA class
  \KOMAoptions{parskip=half}}
\makeatother
\usepackage{xcolor}
\definecolor{default-linkcolor}{HTML}{A50000}
\definecolor{default-filecolor}{HTML}{A50000}
\definecolor{default-citecolor}{HTML}{4077C0}
\definecolor{default-urlcolor}{HTML}{4077C0}
% load footmisc in order to customize footnotes (footmisc has to be loaded before hyperref, cf. https://tex.stackexchange.com/a/169124/144087)
\usepackage[hang,flushmargin,bottom,multiple]{footmisc}
\setlength{\footnotemargin}{0.8em} % set space between footnote nr and text
\setlength{\footnotesep}{\baselineskip} % set space between multiple footnotes
\setlength{\skip\footins}{0.3cm} % set space between page content and footnote
\setlength{\footskip}{0.9cm} % set space between footnote and page bottom
\usepackage[margin=2.5cm,includehead=true,includefoot=true,centering,]{geometry}
\usepackage{longtable,booktabs,array}
\usepackage{calc} % for calculating minipage widths
% Correct order of tables after \paragraph or \subparagraph
\usepackage{etoolbox}
\makeatletter
\patchcmd\longtable{\par}{\if@noskipsec\mbox{}\fi\par}{}{}
\makeatother
% Allow footnotes in longtable head/foot
\IfFileExists{footnotehyper.sty}{\usepackage{footnotehyper}}{\usepackage{footnote}}
\makesavenoteenv{longtable}
% add backlinks to footnote references, cf. https://tex.stackexchange.com/questions/302266/make-footnote-clickable-both-ways
\usepackage{footnotebackref}
\usepackage{graphicx}
\makeatletter
\newsavebox\pandoc@box
\newcommand*\pandocbounded[1]{% scales image to fit in text height/width
  \sbox\pandoc@box{#1}%
  \Gscale@div\@tempa{\textheight}{\dimexpr\ht\pandoc@box+\dp\pandoc@box\relax}%
  \Gscale@div\@tempb{\linewidth}{\wd\pandoc@box}%
  \ifdim\@tempb\p@<\@tempa\p@\let\@tempa\@tempb\fi% select the smaller of both
  \ifdim\@tempa\p@<\p@\scalebox{\@tempa}{\usebox\pandoc@box}%
  \else\usebox{\pandoc@box}%
  \fi%
}
% Set default figure placement to htbp
% Make use of float-package and set default placement for figures to H.
% The option H means 'PUT IT HERE' (as  opposed to the standard h option which means 'You may put it here if you like').
\usepackage{float}
\floatplacement{figure}{H}
\makeatother
\setlength{\emergencystretch}{3em} % prevent overfull lines
\providecommand{\tightlist}{%
  \setlength{\itemsep}{0pt}\setlength{\parskip}{0pt}}
\setcounter{secnumdepth}{5}
% Make \paragraph and \subparagraph free-standing
\makeatletter
\ifx\paragraph\undefined\else
  \let\oldparagraph\paragraph
  \renewcommand{\paragraph}{
    \@ifstar
      \xxxParagraphStar
      \xxxParagraphNoStar
  }
  \newcommand{\xxxParagraphStar}[1]{\oldparagraph*{#1}\mbox{}}
  \newcommand{\xxxParagraphNoStar}[1]{\oldparagraph{#1}\mbox{}}
\fi
\ifx\subparagraph\undefined\else
  \let\oldsubparagraph\subparagraph
  \renewcommand{\subparagraph}{
    \@ifstar
      \xxxSubParagraphStar
      \xxxSubParagraphNoStar
  }
  \newcommand{\xxxSubParagraphStar}[1]{\oldsubparagraph*{#1}\mbox{}}
  \newcommand{\xxxSubParagraphNoStar}[1]{\oldsubparagraph{#1}\mbox{}}
\fi
\makeatother
\ifLuaTeX
\usepackage[bidi=basic]{babel}
\else
\usepackage[bidi=default]{babel}
\fi
\babelprovide[main,import]{brazilian}
% get rid of language-specific shorthands (see #6817):
\let\LanguageShortHands\languageshorthands
\def\languageshorthands#1{}
\makeatletter
\@ifpackageloaded{caption}{}{\usepackage{caption}}
\AtBeginDocument{%
\ifdefined\contentsname
  \renewcommand*\contentsname{Índice}
\else
  \newcommand\contentsname{Índice}
\fi
\ifdefined\listfigurename
  \renewcommand*\listfigurename{Lista de Figuras}
\else
  \newcommand\listfigurename{Lista de Figuras}
\fi
\ifdefined\listtablename
  \renewcommand*\listtablename{Lista de Tabelas}
\else
  \newcommand\listtablename{Lista de Tabelas}
\fi
\ifdefined\figurename
  \renewcommand*\figurename{Figura}
\else
  \newcommand\figurename{Figura}
\fi
\ifdefined\tablename
  \renewcommand*\tablename{Tabela}
\else
  \newcommand\tablename{Tabela}
\fi
}
\@ifpackageloaded{float}{}{\usepackage{float}}
\floatstyle{ruled}
\@ifundefined{c@chapter}{\newfloat{codelisting}{h}{lop}}{\newfloat{codelisting}{h}{lop}[chapter]}
\floatname{codelisting}{Listagem}
\newcommand*\listoflistings{\listof{codelisting}{Lista de Listagens}}
\makeatother
\makeatletter
\makeatother
\makeatletter
\@ifpackageloaded{caption}{}{\usepackage{caption}}
\@ifpackageloaded{subcaption}{}{\usepackage{subcaption}}
\makeatother
\usepackage{bookmark}
\IfFileExists{xurl.sty}{\usepackage{xurl}}{} % add URL line breaks if available
\urlstyle{same}
\hypersetup{
  pdftitle={Filosofia Prática e Ceticismo Moral: Sobre Kant e Hume},
  pdfauthor={Paulo Roberto Martins Cunha},
  pdflang={pt-Br},
  pdfsubject={Filosofia Moral},
  pdfkeywords={Kant, Hume, Filosofia Prática, Ceticismo Moral},
  colorlinks=true,
  linkcolor={blue},
  filecolor={default-filecolor},
  citecolor={default-citecolor},
  urlcolor={default-urlcolor},
  breaklinks=true,
  pdfcreator={LaTeX via pandoc with the Eisvogel template}}
\title{Filosofia Prática e Ceticismo Moral: Sobre Kant e Hume}
\usepackage{etoolbox}
\makeatletter
\providecommand{\subtitle}[1]{% add subtitle to \maketitle
  \apptocmd{\@title}{\par {\large #1 \par}}{}{}
}
\makeatother
\subtitle{Mestrado em Filosofia - Projeto de Pesquisa}
\author{Paulo Roberto Martins Cunha}
\date{8 de maio de 2025}



%%
%% added
%%


%
% for the background color of the title page
%
\usepackage{pagecolor}
\usepackage{afterpage}
\usepackage[margin=2.5cm,includehead=true,includefoot=true,centering]{geometry}

%
% break urls
%
\PassOptionsToPackage{hyphens}{url}

%
% When using babel or polyglossia with biblatex, loading csquotes is recommended
% to ensure that quoted texts are typeset according to the rules of your main language.
%
\usepackage{csquotes}

%
% captions
%
\definecolor{caption-color}{HTML}{777777}
\usepackage[font={stretch=1.2}, textfont={color=caption-color}, position=top, skip=4mm, labelfont=bf, singlelinecheck=false, justification=centering]{caption}
\setcapindent{0em}

%
% blockquote
%
\definecolor{blockquote-border}{RGB}{221,221,221}
\definecolor{blockquote-text}{RGB}{119,119,119}
\usepackage{mdframed}
\newmdenv[rightline=false,bottomline=false,topline=false,linewidth=3pt,linecolor=blockquote-border,skipabove=\parskip]{customblockquote}
\renewenvironment{quote}{\begin{customblockquote}\list{}{\rightmargin=0em\leftmargin=0em}%
\item\relax\color{blockquote-text}\ignorespaces}{\unskip\unskip\endlist\end{customblockquote}}

%
% Source Sans Pro as the default font family
% Source Code Pro for monospace text
%
% 'default' option sets the default
% font family to Source Sans Pro, not \sfdefault.
%
\ifnum 0\ifxetex 1\fi\ifluatex 1\fi=0 % if pdftex
    \usepackage[default]{sourcesanspro}
  \usepackage{sourcecodepro}
  \else % if not pdftex
    \usepackage[default]{sourcesanspro}
  \usepackage{sourcecodepro}

  % XeLaTeX specific adjustments for straight quotes: https://tex.stackexchange.com/a/354887
  % This issue is already fixed (see https://github.com/silkeh/latex-sourcecodepro/pull/5) but the
  % fix is still unreleased.
  % TODO: Remove this workaround when the new version of sourcecodepro is released on CTAN.
  \ifxetex
    \makeatletter
    \defaultfontfeatures[\ttfamily]
      { Numbers   = \sourcecodepro@figurestyle,
        Scale     = \SourceCodePro@scale,
        Extension = .otf }
    \setmonofont
      [ UprightFont    = *-\sourcecodepro@regstyle,
        ItalicFont     = *-\sourcecodepro@regstyle It,
        BoldFont       = *-\sourcecodepro@boldstyle,
        BoldItalicFont = *-\sourcecodepro@boldstyle It ]
      {SourceCodePro}
    \makeatother
  \fi
  \fi

%
% heading color
%
\definecolor{heading-color}{RGB}{40,40,40}
\addtokomafont{section}{\color{heading-color}}
% When using the classes report, scrreprt, book,
% scrbook or memoir, uncomment the following line.
%\addtokomafont{chapter}{\color{heading-color}}

%
% variables for title, author and date
%
\usepackage{titling}
\title{Filosofia Prática e Ceticismo Moral: Sobre Kant e Hume}
\author{Paulo Roberto Martins Cunha}
\date{8 de maio de 2025}

%
% tables
%

\definecolor{table-row-color}{HTML}{F5F5F5}
\definecolor{table-rule-color}{HTML}{999999}

%\arrayrulecolor{black!40}
\arrayrulecolor{table-rule-color}     % color of \toprule, \midrule, \bottomrule
\setlength\heavyrulewidth{0.3ex}      % thickness of \toprule, \bottomrule
\renewcommand{\arraystretch}{1.3}     % spacing (padding)


%
% remove paragraph indentation
%
\setlength{\parindent}{0pt}
\setlength{\parskip}{6pt plus 2pt minus 1pt}
\setlength{\emergencystretch}{3em}  % prevent overfull lines

%
%
% Listings
%
%


%
% header and footer
%
\usepackage[headsepline,footsepline]{scrlayer-scrpage}

\newpairofpagestyles{eisvogel-header-footer}{
  \clearpairofpagestyles
  \ihead*{Mestrado em Filosofia}
  \chead*{}
  \ohead*{Filosofia Prática e Ceticismo Moral: Sobre Kant e Hume}
  \ifoot*{Projeto de Pesquisa}
  \cfoot*{}
  \ofoot*{\thepage}
  \addtokomafont{pageheadfoot}{\upshape}
}
\pagestyle{eisvogel-header-footer}

\deftripstyle{ChapterStyle}{}{}{}{}{\pagemark}{}
\renewcommand*{\chapterpagestyle}{ChapterStyle}


%%
%% end added
%%

\begin{document}

%%
%% begin titlepage
%%
\begin{titlepage}
\newgeometry{left=6cm}
\newcommand{\colorRule}[3][black]{\textcolor[HTML]{#1}{\rule{#2}{#3}}}
\begin{flushleft}
\noindent
\\[-1em]
\color[HTML]{5F5F5F}
\makebox[0pt][l]{\colorRule[435488]{1.3\textwidth}{4pt}}
\par
\noindent

{
  \setstretch{1.4}
  \vfill
  \noindent {\huge \textbf{\textsf{Filosofia Prática e Ceticismo Moral:
Sobre Kant e Hume}}}
    \vskip 1em
  {\Large \textsf{Mestrado em Filosofia - Projeto de Pesquisa}}
    \vskip 2em
  \noindent {\Large \textsf{Paulo Roberto Martins Cunha}}
  \vfill
}


\textsf{8 de maio de 2025}
\end{flushleft}
\end{titlepage}
\restoregeometry
\pagenumbering{arabic}

%%
%% end titlepage
%%

% \maketitle

\setcounter{chapter}{1}
\addtocounter{chapter}{-1}

\renewcommand*\contentsname{Índice}
{
\hypersetup{linkcolor=}
\setcounter{tocdepth}{3}
\tableofcontents
\newpage
}
\section{Projeto de Pesquisa}\label{projeto-de-pesquisa}

\subsection{Delimitação do tema}\label{delimitauxe7uxe3o-do-tema}

Na Segunda Secção da Fundamentação da metafísica dos costumes, em que se
propõe a fazer a passagem do pensamento moral popular à metafísica dos
costumes, Kant faz uma constatação que parece atingir em cheio toda
ideia de metafísica. Ele nos diz com sinceridade:

\begin{quote}
Na realidade, é absolutamente impossível encontrar na experiência com
perfeita certeza um único caso em que a máxima de uma ação, de resto
conforme ao dever, se tenha baseado puramente em motivos morais e na
representação do dever'' (Kant, p.~40)
\end{quote}

Essa é uma colocação acerca da nossa experiência moral que, em
princípio, é a expressão clara da impossibilidade, para os homens, de
agirem segundo princípios morais gerais, embora saibamos que a intenção
de Kant com a sua Fundamentação é encontrar justamente algo que
desempenhe essa função, servindo de critério puro e necessário para toda
e qualquer avaliação do comportamento moral. Na Fundamentação, Kant
chamará a esse princípio puro de ``autonomia da vontade''.

Ocorre que o que expressa o filósofo na passagem citada é,
indiscutivelmente, a mesma constatação que está na base de todo
ceticismo moral. Uma das razões, só para tomar um exemplo, seria o
desentendimento reinante acerca do que se quer dizer quando o assunto é
a moral ou a ética. Adauto Novaes, em um livro de múltiplos autores por
ele organizado, nos diz que existem várias teorias sobre o tema porque:
``A palavra ética, por exemplo, não tem o mesmo sentido para todos''
(Novaes, 2007, p.~8). E seria essa falta ou mesmo impossibilidade de se
chegar a algum consenso a respeito do comportamento moral que produziria
em nós quase que naturalmente o que chamaremos aqui de ceticismo moral,
para que não se confunda, inicialmente, com o ceticismo epistemológico
ou cognitivo. Mas é o cético Hume, que desconfiava da pureza do
entendimento humano, quem escreve, em um tom de admiração:

\begin{quote}
Para alguns filósofos parece constituir motivo de surpresa que, sendo
todos os homens dotados da mesma natureza e possuidores das mesmas
faculdades, apesar disso sejam tão imensamente diferentes em suas ações
e tendências, e que uns condenem intransigentemente o que outros
prazerosamente procuram (Hume, 1980, p.~215)
\end{quote}

Hume se admira que ainda se encontre aqui alguma surpresa, afinal, um
mesmo homem pode ter comportamentos contraditórios, como se nunca
seguisse, para si mesmo, as mesmas regras ao agir. Para Hume isso é uma
constatação, até porque a história humana é o registro de fatos e
narrativas de ações que põem em questão a ideia de que os homens possam
um dia encontrar um guia seguro para suas ações, como aparentemente
fizeram com seus juízos científicos. O fato é que o homem real,
existente, não é como o cientista real, como um Newton, por exemplo.

Mesmo assim, sabendo que a Física tem uma parte empírica e real, mas,
também, uma parte racional, Kant se pergunta se as ações humanas não
poderiam ser estudas separando a parte empírica da parte pura. Pois
assim poderíamos distinguir nossa reflexão sobre o homem em suas ações
pragmáticas, que Kant chama de ``Antropologia prática'', das reflexões
voltadas às exigências feitas às ações e que não se explicam pelo seu
efeito imediato. E é a essa última forma de reflexão que Kant chamará de
``Moral''. Com isso, poderíamos nos guiar não pelo que sabemos da
história humana e dos feitos dos homens, mas o que a razão em nós nos
diz, constantemente, que deveríamos fazer, mesmo que, pelos mais
variados motivos, não o consigamos realizar.

Por isso, o que estamos proponde analisar nesta dissertação é o
seguinte: qual a solução kantiana para o problema do ceticismo moral, e
qual o grau de satisfação podemos reconhecer no modo como ele trata a
questão, na medida em que ele mesmo reconhece que, na realidade, a
teoria é outra coisa? Por que ele se pergunta: ``Não é verdade que é da
mais extrema necessidade elaborar um dia uma pura Filosofia moral que
seja completamente depurada de tudo o que possa ser somente empírico e
pertença à Antropologia?'' (Kant, 1974, p.~198).

\subsection{Problema}\label{problema}

\subsection{Justificativa}\label{justificativa}

\subsection{Objetivos}\label{objetivos}

\begin{enumerate}
\def\labelenumi{\arabic{enumi}.}
\tightlist
\item
  \textbf{Gerais}: Expor a solução kantiana ao problema do ceticismo
  moral.
\item
  \textbf{Específicos}: 1. Apresentar a filosofia crítica como uma
  reação positiva ao ceticismo moderno, em especial no que tange sua
  perspectiva moral; 2. Recensear, com o auxílio da obra de Hume, alguns
  pontos fundamentais para a contestação da orientação racionalista e
  idealista da moralidade; 3. Mostrar que a resposta kantiana ao
  ``´problema de Hume'', além de seu alcance teórico, contém a chave
  para a possibilidade lógica da moral.
\end{enumerate}

\subsection{Metodologia}\label{metodologia}

\subsection{Cronograma}\label{cronograma}

Este projeto de pesquisa de mestrado está previsto para ser desenvolvido
inicialmente num período de dois anos. De acordo com o tema escolhido e
os objetivos propostos, esperíodo inicial deve ser dividido em quatro
sub-períodos de pesquisa, que correspondem a quatro semestres, de modo a
compreender os seguintes campos de investigação:

\begin{itemize}
\tightlist
\item
  \textbf{1º semestre}: Além de cursar as disciplinas para obtenção dos
  créditos, fazer um levantamento de novos títulos e revisão
  bibliográfica. Com isso cremos poder reforçar e consolidar a estrutura
  do projeto.
\item
  \textbf{2º semestre}: Pretendemos dar continuidade ao cumprimento dos
  créditos, se necessário, e estruturar a construção do primeiro
  capítulo da Dissertação, cujo objetivo é apresentar os fundamentos
  para a nossa abordagem do pensamento de Rousseau e dos primeiros
  românticos.
\item
  \textbf{3º semestre}: Nossa intenção é a de reunir, sobretudo no
  Discurso sobre as ciências e as artes e no Emílio, os elementos para
  uma dissertação sobre as noções de subjetividade e sentimento em
  Rousseau.
\item
  \textbf{4º semestre}: Neste momento nossa atenção deverá ser dirigida
  à seleção de fragmentos de Novalis e Schlegel nos quais possamos
  reconhecer o reflexo, por assim dizer, de uma concordância de
  percepção do romantismo alemão com o que prenuncia Rousseau em suas
  obras, procurando corrigir as distorções comuns com as quais se tenta
  associar esses dois momentos da história da filosofia moderna mediados
  pela filosofia crítica. Até o final desses quatro semestres
  acreditamos poder reunir material suficiente para dar conta dos nossos
  objetivos.
\end{itemize}

\subsection*{Bibliografia}\label{bibliografia}
\addcontentsline{toc}{subsection}{Bibliografia}

\end{document}
